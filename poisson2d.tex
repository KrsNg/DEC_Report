\documentclass[a4paper, 10pt]{article}
%
\usepackage{amsmath, amsthm, amssymb, url}
\usepackage{mathrsfs}           % pour \mathscr
\usepackage[latin1]{inputenc}
\usepackage[colorlinks=true]{hyperref}
\usepackage{xspace}     % Vivement conseill� lorsqu'on utilise babel avec
                        % l'option � frenchb � (xspace permet une meilleure
                        % gestion de \og et \fg pour les guillemets)

\usepackage{graphicx} % Si le document doit inclure des images
\usepackage{epsfig}
\usepackage{makeidx}
\usepackage{palatino}
% \usepackage{algorithm} % Caution : needs to be a the end of the package list
% \usepackage{algorithmic}

% \textwidth 15truecm \textheight 22truecm
\setlength{\textwidth}{6.8in}
\setlength{\textheight}{9.4in}

\usepackage{qresize}
\resize

\parskip 0.05in

%\setcounter{secnumdepth}{-2}

%\newcommand{\half}{{\frac{1}{2}}}
%
\newtheorem{assumption}{Assumption}
\newtheorem{hypothesis}{Hypothesis}
\newtheorem{theorem}{Theorem}
\newtheorem{defi}{Definition}
\newtheorem{prop}{Proposition}
\newtheorem{lem}{Lemma}
\newtheorem{rem}{Remark}
\newcommand{\bx}{\mathbf{x}}
\newcommand{\by}{\mathbf{y}}
\newcommand{\bu}{\mathbf{u}}
\newcommand{\bg}{\mathbf{g}}
\newcommand{\be}{\mathbf{e}}
\newcommand{\bv}{\mathbf{v}}
\newcommand{\bz}{\mathbf{z}}
\newcommand{\buu}{\mathbf{U}}
\newcommand{\bV}{\mathbf{V}}
\newcommand{\bR}{\mathbf{R}}
\newcommand{\bff}{\mathbf{f}}
\newcommand{\bX}{\mathbf{X}}
\newcommand{\bA}{\mathbf{A}}
\newcommand{\bK}{\mathbf{K}}
\newcommand{\bI}{\mathbf{I}}
\newcommand{\bE}{\mathbf{E}}
\newcommand{\bB}{\mathbf{B}}
\newcommand{\bL}{\mathbf{L}}
\newcommand{\bb}{\mathbf{b}}
\newcommand{\bc}{\mathbf{c}}
\newcommand{\bM}{\mathbf{M}}
\newcommand{\br}{\mathbf{r}}
\newcommand{\btheta}{\boldsymbol{\theta}}
\newcommand{\bbeta}{\boldsymbol{\beta}}
\newcommand{\bzero}{\mathbf{0}}
\newcommand{\bvarphi}{\boldsymbol{\varphi}}
\newcommand{\bPhi}{\boldsymbol{\Phi}}
\newcommand{\balpha}{\boldsymbol{\alpha}}
\newcommand{\bxi}{\boldsymbol{\xi}}
\newcommand{\bgamma}{\boldsymbol{\gamma}}
\newcommand{\btau}{\boldsymbol{\tau}}
\newcommand{\bnabla}{\boldsymbol{\nabla}}
\newcommand{\uth}{u^{\btheta}}
\newcommand{\fth}{f^{\btheta}}
\newcommand{\gth}{g^{\btheta}}
\newcommand{\vth}{v^{\btheta}}
\newcommand{\wth}{w^{\btheta}}
\newcommand{\zth}{z^{\btheta}}
\newcommand{\ba}{\mathbf{a}}
\newcommand{\bw}{\mathbf{w}}
\newcommand{\etk}{\eta^k}
\newcommand{\xik}{\xi^k}
\newcommand{\bbth}{{\mathbf{b}}^{\theta_3}}
\newcommand{\xiav}{\langle \bxi \rangle}

\reversemarginpar
%\al

%/////////////////////////////////////////////////
\begin{document}
%/////////////////////////////////////////////////

\begin{center}
\Large Two-dimensional Poisson equation
\end{center}

\vspace{1cm}

We seek to solve the Poisson equation
\begin{eqnarray}
-\Delta u &=& f \mbox{ in } D,\label{eq:1}\\[1ex]
%
u &=& 0 \mbox{ on } \partial D,\label{eq:2}
\end{eqnarray}
where $D=[0,1]^2$, $f(x,y)=2(x(1-x)+y(1-y))$ is the source term and $u(x,y)$ is the solution to be computed.

\begin{enumerate}

\item We consider a finite difference (FD) scheme for solving (\ref{eq:1}). The spatial domain $D$ is discretized using a
regular grid made of $(N+1)^2$ points $\bx_{ij}=(ih,jh)$ with $0\leq i,j \leq N$, $h=\frac{1}{N}$. We denote by $u_{ij}$
and $f_{ij}$ the approximate solution $u(\bx_{ij})$ and $f(\bx_{ij})$, respectively.
Using the centered FD scheme
$$
-\frac{u_{i+1,j}+u_{i-1,j}+u_{i,j+1}+u_{i,j-1}-4 u_{ij}}{h^2}=f_{ij},
$$
written at each interior node $\bx_{ij}$, $1 \leq i,j \leq N-1$, and taking into account the boundary conditions
(\ref{eq:2}), write down a linear system 
\begin{equation}\label{eq:linsyst}
\bA \bu = \bb
\end{equation}
of size $(N-1)^2\times (N-1)^2$, where $\bu$ is a vector that contains the $u_{ij}$ corresponding to the interior nodes.

\item Code up the resolution of (\ref{eq:linsyst}) with $N=500$ using Trilinos 
(assemble $\bA$, $\bb$ and get $\bu$ using GMRES algorithms).

\item Validation: Check the approximate solution corresponds to the exact solution $u_{ex}(x,y)=xy(1-x)(1-y)$. 
Plot in Matlab the approximate/exact solutions and compute the error norm $||u_{approx}-u_{ex}||_{L^2}$, where $||\cdot||_{L^2}$
is computed as $||u||_{L^2}=\int_{D} u^2(\bx)d\bx \simeq h \bigl(\sum_{i,j}u^2_{ij}\bigr)^{1/2}$.

\item Use a sequential implementation (for which global and local indices are the same) and a parallel version based on local indices.

\end{enumerate}



%/////////////////////////////////////////////////
\end{document}
%/////////////////////////////////////////////////