\section{Introduction and motivation}

Computational Fluid Dynamics (CFD) has been developed to help reduce time and cost of prototype design for engineering systems. Practical reactive flows almost always involve turbulence, and CFD methods have been developed to help capture such complex phenomena to varying extents of accuracy. Corresponding physical experiments can take a long time and can be very expensive to conduct. Three main approaches exist. Reynolds averaged Navier Stokes (RANS) uses full modeling of turbulent scales. Large eddy simulations (LES) model sub-filter scales (SFS) while resolving larger scales. Direct numerical simulations (DNS) resolve all the turbulent scales. Supercomputers have been used within academic and research lab collaborations for the DNS of turbulence in a box such as that by Kaneda and Ishihara ~\cite{kaneda:2006}, but such calculations are very expensive and limited to simple cases with low Reynolds numbers. More complicated analyses such as those involving combustion simulations use LES to achieve an overall good prediction of results at an accuracy greater than RANS, but at a cost lower than DNS.\par 

To improve accuracy within the CFD models, we need to reduce numerical error within LES simulations. Key sources of error in CFD include numerical and modeling errors. Examples of numerical errors are solution errors, truncation errors and convergence errors. A common alternative for reducing error is the gradient (physics)-based approach, but this technique is actually less efficient, in that it error reduction is not guaranteed for continual levels of mesh ($h$) refinement. More on gradient-based methods will be discussed in section \ref{section:Adjoint}. This work proposes an implementation of an adjoint-based technique for error estimation. The adjoint-based method is a more effective approach to reduce numerical errors at a potentially lower cost would be the use of adjoint-based error estimates for localized error reduction. The adjoint-based technique is frequently used for localized mesh adaptation ($h$) refinement. However, this work proposes the combination of ($h$) refinement with ($p$) refinement. That is, the additional use of high-order discretization schemes (of accuracy $ p>2 $) to further improve accuracy of the CFD results, while lowering the cost to attain them.  This combination can be expressed as an adjoint-based $\mathcal{O}(h^p)$ error estimation technique and is proposed for error reduction (while reducing computational costs) in LES of reactive flow simulations.\par 