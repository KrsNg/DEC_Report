\subsection{Finite volume method (FVM)}
Finite volume methods (FVM) typically apply a cell-centered spatial discretization. 

In such an approach, the solution domain is divided into
discrete control volumes~\cite{Toro:1997} and the integral form of the governing equations are solved in each element or cell. The Favre-filtered form of the Navier Stokes equations are of primary interest here and will be used for the prediction of turbulent reactive flows.  They are described
in greater detail in the appendix (see Sections~\ref{subsection:Navier} and~\ref{subsection:Favre}).
The weak conservation form of the Favre-filtered Navier-Stokes (FANS) equations can be expressed using matrix-vector notation~\cite{Neto:2014} as:
%%
\begin{equation}
\label{eq:weak.form}
\frac{\partial \overline{\Vector{U}}}{\partial t} 
+ \vec{\nabla} \cdot \vec{\mathcal{F}} =
\frac{\partial \overline{\Vector{U}}}{\partial t} + 
\vec{\nabla} \cdot \vec{\mathcal{F}}^{\mathrm{I}} \left( \overline{\Vector{U}} \right) - 
\vec{\nabla} \cdot \vec{\mathcal{F}}^{\mathrm{V}} 
\left( \overline{\Vector{U}}, \vec{\nabla} \overline{\Vector{U}} \right) = 
\overline{\Vector{S}}
\end{equation}
%%
where:
\begin{itemize}
 \item $\overline{\Vector{U}}$ is the vector of conserved solution variables, and 
 \item $\vec{\mathcal{F}}$ is the solution flux dyad.
\end{itemize}  
 
The flux dyad can further be decomposed into two components and written as:
\begin{itemize}
\item $\vec{\mathcal{F}} =  \vec{\mathcal{F}}^{\mathrm I} - \vec{\mathcal{F}}^{\mathrm{V}}$ where $\vec{\mathcal{F}}^{\mathrm I} = \vec{\mathcal{F}}^{\mathrm I}(\overline{\Vector{U}})$ which contains the hyperbolic or inviscid components of the solution fluxes, and 
\item $\vec{\mathcal{F}}^{\mathrm{V}} =  \vec{\mathcal{F}}^{\mathrm{V}}(\overline{\Vector{U}}, \vec{\nabla} \overline{\Vector{U}})$ which contains the elliptic or viscous components of the fluxes.
\end{itemize}

\noindent The inviscid fluxes, the hyperbolic part, at the face of each computational cell or control volume are resolved via the solution to a Riemann Problem, given the different left and right states for each cell. Godunov's paper~\cite{godunov:1959} discusses ways to evaluate solutions to the Riemann problem. The goal is to obtain the final cell-centered, cell-average value. Typically, approximate Riemann solvers are used to this end. The viscous fluxes (elliptic components of the flux) at the
cell boundaries are evaluated using centrally weighted
schemes. See Gao et al~\cite{gao:2006a, gao:2011}

For the purpose of CFD, these governing equations in PDE form need to be discretized. The integral form of Eq.~(\ref{eq:weak.form}) is applied to a hexahedral reference cell $(i,j,k)$ and an $N_G$-point Gaussian quadrature numerical integration procedure is used to evaluate the solution flux along each of the $N_f$ faces of the cell, to obtain:

\begin{equation}
\label{eqn:semi_discrete}
 \frac{\mathrm{d}\overline{\Vector U}_{ijk}}{\mathrm{d}t} =
 - \frac{1}{{V}_{ijk}} \sum_{l=1}^{N_f} 
 \sum_{m=1}^{N_{GF}} \left( \omega_m \left(\vec{\mathcal{F}^{\mathrm{I}}} - 
 \vec{\mathcal{F}^{\mathrm{V}}} \right)\cdot \hat{n} {A} \right)_{ijk,l,m}
 + \sum_{n=1}^{N_{GV}} \left( \omega_n {\Vector S} \right)_{i,j,k,n}
 = \overline{\Vector{R}}_{ijk} \left( \overline{\Vector{U}} \right),
\end{equation}
where $\omega_m$ are the face quadrature weighting coefficients, $\omega_n$ are the volumetric quadrature weighting coefficients, $A_l$ denotes the surface area of face $l$, and $\overline{\Vector{R}}_{ijk}$ is the residual operator. Quadrature rules are used to determine the flux evaluation points on the surface, $N_{GF}$ and those for the volume, $N_{GV}$. The remaining solution procedure is as follows:
\begin{itemize}
\setlength\itemsep{0.1em}
 \item Apply Solution Reconstruction (see Section~\ref{section:CENO});
 \item Evaluate the Inviscid and Viscous Fluxes and apply the weights to respective Gauss-Legendre points;
 \item Evaluate the Source Vector, which adds effects of turbulence and chemistry in reacting flows;
 \item Apply an appropriate time-marching scheme, such as a fourth-order Runge-Kutta (\textbf{RK4}) suitable for high-order methods.
\end{itemize}

\subsubsection{Central essentially non-oscillatory (CENO) scheme}
\label{section:CENO}
In solution reconstruction, given cell average values, a Taylor series expansion polynomial is defined to represent the variation of the solution within the particular cell and the average solution of its neighbors. 
The high order CENO reconstruction scheme of Ivan and Groth~\cite{ivan:2007, ivan:2013b} uses a \textit{k}-exact least-squares reconstruction technique, essentially a $k^{th}$ order Taylor series expansion of solution variable $U$ about the cell center. This technique is applied to a monotonicity-preserving limited linear scheme. For solution of the FANS equations having viscous terms, \textit{k}
corresponds directly to the spatial accuracy of the scheme. A smoothness indicator is used to switch between reconstruction procedures.\par
Some of the advantages of CENO are~\cite{Groth:2013}:
\begin{itemize}
\setlength\itemsep{0.1em}
\item Provides accuracy of other essentially non-oscillatory (ENO) schemes while maintaining monotonicity near discontinuities;
\item Identifies regions where under-resolution and non-smooth data occurs, and this could prove useful for mesh adaptation techniques.
\end{itemize}
