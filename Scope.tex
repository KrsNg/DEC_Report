\section{Scope of Research}
This work proposes a two-pronged technique to reduce the numerical error arising from discretization of both the governing equations and the computational domain (mesh).\par

This will involve implementation of High Order Finite Volume Scheme, as well as Adjoint-Based Error Estimation method, used to compute sensitivity of the output to a given input, and the results will be used for mesh refinement. Presently, error reduction is done by use of gradient-based error indicators used in combination with High-Order methods such as the Finite Volume Method, or Finite Element Method. More on Gradient-Based methods will be discussed in Section 4. \par

The approach to be used will be to first implement the Adjoint solver within the group CFFC Code as modified by Northrup \cite{Northrup:2013}, and validate it on already known inviscid (Euler) laminar steady solutions, and then implement it for unsteady cases, and then introduce viscosity and the temporal variation. Once the Adjoint solver works, the next step will be to advance the work done by Deconinck to test and implement appropriate LES Explicit filters \cite{Deconinck:2008} to the High-Order Finite Volume Scheme which, for Turbulent Premixed flames, uses the Favre-Averaged Navier Stokes (FANS) equations and the Central Essentially Non-Oscillatory Scheme (CENO). The reasoning behind this is that a higher order of discretization provides an improved accuracy of the solution. Combining this with Adjoint-Based Error Estimation will allow for an optimal technique for mesh refinement. The computational cost versus benefits (improved accuracy, ect) is not known at the moment - this will be a later conclusion at the end of this investigation. Ultimately, comparisons will be run to test the prediction of the pressure oscillations for High-Order and standard Second-Order schemes.\par

The CFFC Code already has most of the needed functionality required for this work. Ivan and Groth implemented the Central Essentially Non-Oscillatory Scheme \cite{ivan:2007b}, Two-Dimensional Block-Based Adaptive Mesh Refinement has been materialized by Zhang and Groth \cite{Zhang:2011a}, and extended to Three Dimensions by Williamschen and Groth \cite{Williamschen:2013}. Northrup and Groth \cite{Northrup:2013} implemented a Newton Krylov Schwarz Solver, namely GMRES, for the CFFC Code, and thus relieved the Program from the CFL condition constraints.\par

The application of interest is the study of Thermoacoustic phenomenon within the model Turbomeca Combustor. This work will then attempt to compare the effect of increased accuracy (due to using Higher-Order Discretization and the Adjoint-Based Error Estimation for Mesh Refinement) on predictions of the onset of Heat and Pressure feedback cycles which can easily cause damage to the components of the combustor. Results will be compared with experimental analysis as performed by Brisebois \textit{et al} \cite{Brisebois:2014}. \par
