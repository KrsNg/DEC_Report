\section{Introduction and Motivation}

Computational Fluid Dynamics (CFD) has been developed to reduce the time and cost of prototypes in fluid flow experiments. Typical product life-cycles from conception to testing involve numerous design iterations and modifications. From the mid 80s to 90s the stability and reliability of CFD algorithms was improved and CFD's present role has grown to include a vast spectrum of modern day industries. Gordon Moore predicted that computing power would double approximately every two years.~\cite{intel:2005}. Modern day massively-paralleled computer systems have brought computing power to the peta- and exa-scale levels.\par

Real fluid flows almost always involve turbulence, and CFD methods and models have been developed to capture this phenomenon to varying extents of accuracy. Three main approaches exist: Reynolds Averaged Navier Stokes (RANS), which uses full modeling of turbulence phenomenon; Large Eddy Simulations (LES) do direct solutions of Turbulence up to some intermediate Taylor scales while modeling the smaller scales; Direct Numerical Simulation (DNS) performs full resolution and calculation of the turbulence without any modelling. Supercomputers have been used within academic and research lab collaborations for the Direct Numerical Simulation (DNS) of "Turbulence in a Box" such as that by Kaneda and Ishihara. ~\cite{kaneda:2006}, but such calculations are limited to low Reynolds numbers and for simple cases. For more complicated analysis, LES is more practical and achievable, such as those involving Combustion simulations.\par 

The ultimate validation will be for Thermoacoustic analysis  for a Turbomeca Model Combustor. Brisebois \textit{et al} did Thermoacoustic analysis on Experimental data for this combustor ~\cite{Brisebois:2014}. Within the combustion chamber, heat release and pressure are coupled; sporadic fluctuations in pressure (can) increase the heat release, which in turn increases the pressure, thus setting up a feedback cycle. If the frequency of pressure oscillations matches the natural frequency of the chamber, the resulting resonance could lead to damage of components in the Combustion chamber, posing huge safety risks and result in very high maintenance costs.\par

Such a simulation would need to be as accurate as possible. To improve accuracy, so that CFD results can be closer to those from experiment, we need to reduce error. Two  key sources of error within CFD include the discretization of the governing equations (or the general numerical scheme) - which are typically composed in the Partial Differential Equation (PDE) form, and the discretization of the geometry, i.e the mesh resolution. Using High-Order discretization schemes Finite Volume Methods will reduce the former, while we choose a novel approach for mesh adaptation - Adjoint-based Error Estimation for Mesh Adaptation. A more common and "cheaper" alternative to mesh refinement criteria is the gradient-based approach, but this technique is actually less efficient, in that it requires new sensitivity evaluations for each new parameter of interest. More on Gradient-Based methods will be discussed in Section 4. The reason for combining high-order methods with error-estimation techniques is that a higher order of discretization provides an improved accuracy of the solution. Combining this with Adjoint-Based Error Estimation will allow for an optimal technique for mesh refinement.\par 