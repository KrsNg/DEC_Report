% ----------------------
% THIS IS MY DEC1 REPORT
% ----------------------

\documentclass[titlepage,11pt,letterpaper]{article}

\usepackage{graphicx,epsfig,color}
\usepackage{subfigure}
\usepackage{setspace}
\usepackage{amsmath,amssymb}
\usepackage[hang,small,bf]{caption}
\usepackage[colorlinks]{hyperref}
\hypersetup{
	colorlinks=false,
	citebordercolor={0 1 0}
}
\setlength{\captionmargin}{10pt}


\newcommand{\pd}[2]{\frac{\partial#1}{\partial#2}}
% Page Margins
\voffset=-1.0in
\topmargin=0.75in
\headheight=0.0in
\headsep=0.0in
\hoffset=-1.0in
\oddsidemargin=0.7in
\evensidemargin=0.7in

\parindent      0.2000in

\textwidth=7.0in
\textheight=9.25in

% Line Spacing
\renewcommand{\baselinestretch}{1.33}

% Alter some LaTeX defaults for better treatment of figures:
% See p.105 of "TeX Unbound" for suggested values.
    % See pp. 199-200 of Lamport's "LaTeX" book for details.
    %   General parameters, for ALL pages:
    \renewcommand{\topfraction}{0.9}	% max fraction of floats at top
    \renewcommand{\bottomfraction}{0.8}	% max fraction of floats at bottom
    %   Parameters for TEXT pages (not float pages):
   \setcounter{topnumber}{2}
    \setcounter{bottomnumber}{2}
    \setcounter{totalnumber}{4}     % 2 may work better
    \setcounter{dbltopnumber}{2}    % for 2-column pages
    \renewcommand{\dbltopfraction}{0.9}	% fit big float above 2-col. text
    \renewcommand{\textfraction}{0.07}	% allow minimal text w. figs
    %   Parameters for FLOAT pages (not text pages):
    \renewcommand{\floatpagefraction}{0.7}	% require fuller float pages
	% N.B.: floatpagefraction MUST be less than topfraction !!
    \renewcommand{\dblfloatpagefraction}{0.7}	% require fuller float pages

    % remember to use [htp] or [htpb] for placement

%\newcommand*{\drop}{\vspace*{0.2\textheight}}



\begin{document}


\newcommand*{\titleRR}
{\begingroup
  \centering
  \vfill
  \begin{figure}
    \vspace{1.5cm}
    \begin{center}
      \includegraphics[width=0.17\textheight]{./figs/utias_logo_blue.jpg}
    \end{center}
    \vspace{1.5cm}
  \end{figure}
  {\LARGE \textbf{Adjoint-Based Error Estimation and Mesh Adaptation for LES of Turbulent Premixed Flames}}\\
  \vspace*{2.0cm}
  {\large Christopher Ngigi}\\
  \vspace*{1.cm}
  {\normalsize\itshape University of
    Toronto Institute for Aerospace Studies}\\
  {\normalsize\itshape 4925
    Dufferin Street, Toronto, Ontario, M3H 5T6, Canada}\\
  \vspace*{2.cm}
  {Doctoral Examination Committee Report} \\
  \vspace*{1.cm}
  {January 2015}
  \vfill\null
\endgroup}

\thispagestyle{empty}\titleRR
\clearpage
\setcounter{page}{1}

%\maketitle
%\tableofcontents
%\newpage

%%%%%%%%%%%%%%%%%%%%%%%%%%%%%%%%%%%%%%%%%%%%%%%%%%%%%%%%%%%%%%%%%%%%%%%%%%%%%%%%%%%%%%%%%%%%
%-------------------------------------------------------------------------------------------
% Introduction and Motivation
%-------------------------------------------------------------------------------------------
%%%%%%%%%%%%%%%%%%%%%%%%%%%%%%%%%%%%%%%%%%%%%%%%%%%%%%%%%%%%%%%%%%%%%%%%%%%%%%%%%%%%%%%%%%%%
\newpage
\section{Introduction and Motivation}
\noindent Computational Fluid Dynamics (CFD) was developed to reduce the cost of fluid flow experiments whose lifecycle spanning from conception to testing involved numerous design iterations and modifications. The initial applications of CFD  focused on aerodynamic and hydrodynamic flows within the engineering industry. At the present time, its role has grown to include a vast spectrum of modern day industries. This is due to the ever-increasing power of computers as predicted by Gordon Moore, namely, that computing power would double approximately every two years.~\cite{intel:2005}\\ \\
\noindent Since the 1960's, the role of CFD in the aerospace industry has grown to involve more complex design configurations and modelling of physics. Quite naturally, this has been extended to aircraft powerplants: Engine performance software was used to design Compressor maps for the purpose of estimating power outputs; Specialty softwares such as \textit{CHEMKIN, CANTERA} are used to calculate combustion rates and products for given reactants and conditions at the lab level; CFD is being applied to simulate the combustion of fuels in laminar and turbulent conditions.\\ \\
\noindent Turbulent combustion is a very complex physics phenomenon. Up to the 90's, researchers were still working on accurate mathematical models that would be capable of predicting to some extent the interaction and behavior of non-laminar fluid flow regimes. Naturally, computers have been employed as tools to simulate these models and results are validated against experiments. \\ \\
\noindent Because of the huge discrepancy in scales when it comes to turbulence, ranging from the Integral scales to the Kolmogorov scales, vast computational resources would be required to model the entire turbulent regime. This high-fidelity branch of CFD is known as Direct Numerical Simulation (DNS), and is only achievable on Supercomputers, such as was recently demonstrated by Kaneda and Ishihara. ~\cite{kaneda:2006} DNS remains very expensive in terms of  computational and time resources: in reality, complex simulations are impossible, some requiring solution time on the order of $10^8$ years.\\ \\
\noindent Large Eddy Simulation(LES) is a branch of CFD has been successfully applied to combustion simulation in the last two decades. It serves as a compromise to model Turbulent scales up to a given specific scale (which we call the filter size): any scale smaller than this is modelled. \\ \\
\noindent Within a Gas Turbine combustion chamber, heat release and pressure are a coupled phenomenon, and any sporadic fluctuations in heat release or pressure will affect the other. This is the field of study for \textit{Thermoacoustics}. Higher temperatures within the combustion chamber raise the pressure which in turn increase the temperature even further, leading to a cycle where heat feeds off pressure and vice versa. It often results in structural damage to components within the combustion chamber and the turbine blades, and these pose huge safety risks and result in very high maintenance costs.\\ \\
\noindent Notice the several stages at which innacuracies can be introduced into the simulation: one is at the algorithm level due to both the meshing of the geometry and the discretization of the governing equations, and the other at the modelling level due to the selection of turbulence models and filter sizes. \\



%%%%%%%%%%%%%%%%%%%%%%%%%%%%%%%%%%%%%%%%%%%%%%%%%%%%%%%%%%%%%%%%%%%%%%%%%%%%%%%%%%%%%%%%%%%%
% -------------------------------------------------------------------------------------------
% Scope of Research
% -------------------------------------------------------------------------------------------
%%%%%%%%%%%%%%%%%%%%%%%%%%%%%%%%%%%%%%%%%%%%%%%%%%%%%%%%%%%%%%%%%%%%%%%%%%%%%%%%%%%%%%%%%%%%
% \newpage
\section{Scope of Research}
\noindent The present work intends to address a two-pronged technique to reduce the computational error arising from discretization and meshing. The technique will involve implementation of the adjoint-based error estimation method to be implemented for mesh refinement. Adjoints will be used for sensitivity analysis, namely, monitoring the computational cells to which the functional outputs of the CFD analysis are most sensitive.\\ \\
\noindent The second part involves the use of high-order finite volume discretization of the governing equations: Favre-Averaged Navier Stokes (FANS). The higher the order of discretization, the higher the accuracy of the solution. This minimizes error and is often desired since for a coarser level of mesh refinement, a higher solution accuracy can be achieved, thus this can be viewed as reducing the computational cost.\\ \\
\noindent As discussed in section 1, it is the intention to ensure that the modifications to the CFFC algorithm will be remain parallelizable, taking advantage of the multiprocessing resources that exist at SCINET. Parallelization is achieved by domain decomposition, such that processors can compute smaller sections of the mesh with the corresponding governing equations all the while communicating and linking to each other through the Message Parsing Interface (MPI). Parallelization is the modern-day approach to increase speed while allowing computationally complex simulations.\\ \\
\noindent The present work will focus on implementing adjoint-based error estimation for mesh refinement and high-order finite volume discretization and attempt to address thermoacoustic phenomenon within turbulent premixed combustion\\ \\




%%%%%%%%%%%%%%%%%%%%%%%%%%%%%%%%%%%%%%%%%%%%%%%%%%%%%%%%%%%%%%%%%%%%%%%%%%%%%%%%%%%%%%%%%%%%
%-------------------------------------------------------------------------------------------
% CFD Techniques to Model Combustion
%-------------------------------------------------------------------------------------------
%%%%%%%%%%%%%%%%%%%%%%%%%%%%%%%%%%%%%%%%%%%%%%%%%%%%%%%%%%%%%%%%%%%%%%%%%%%%%%%%%%%%%%%%%%%%
\newpage
\section{Proposed Adaptive Mesh Refinement Strategy}

\noindent We will preview some of\\
\subsection{The Finite Volume Method}
\subsection{The High-Order Centrally Essentially Non-Oscillatory (CENO) Scheme}
\subsection{Explicit Filtering and Commutation Errors}
\subsection{Block-Based Adaptive Mesh Refinement (AMR)}
\subsubsection{Isotropic AMR}
\subsubsection{Anisotropic AMR}

%%%%%%%%%%%%%%%%%%%%%%%%%%%%%%%%%%%%%%%%%%%%%%%%%%%%%%%%%%%%%%%%%%%%%%%%%%%%%%%%%%%%%%%%%%%%
%-------------------------------------------------------------------------------------------
% Adjoint Method for Error Estimation
%-------------------------------------------------------------------------------------------
%%%%%%%%%%%%%%%%%%%%%%%%%%%%%%%%%%%%%%%%%%%%%%%%%%%%%%%%%%%%%%%%%%%%%%%%%%%%%%%%%%%%%%%%%%%%
\newpage
\section{Adjoint-Based Method for Error Estimation}
\subsection{Introduction}
\subsubsection{Steady Adjoints}
\subsubsection{Unsteady Adjoints}
\subsection{Derivation}
\subsection{Solution of Linear Systems}
\subsection{Use of Solution Error Estimates in Mesh Adaptation}
\subsection{Implementing Isotropic Mesh Refinement}


%%%%%%%%%%%%%%%%%%%%%%%%%%%%%%%%%%%%%%%%%%%%%%%%%%%%%%%%%%%%%%%%%%%%%%%%%%%%%%%%%%%%%%%%%%%%
%-------------------------------------------------------------------------------------------
% Poisson Solver Case
%-------------------------------------------------------------------------------------------
%%%%%%%%%%%%%%%%%%%%%%%%%%%%%%%%%%%%%%%%%%%%%%%%%%%%%%%%%%%%%%%%%%%%%%%%%%%%%%%%%%%%%%%%%%%%
\newpage
\section{Poisson Solver}
\subsection{Model Problem}


%%%%%%%%%%%%%%%%%%%%%%%%%%%%%%%%%%%%%%%%%%%%%%%%%%%%%%%%%%%%%%%%%%%%%%%%%%%%%%%%%%%%%%%%%%%%
%-------------------------------------------------------------------------------------------
% LES of Premixed Flames
%-------------------------------------------------------------------------------------------
%%%%%%%%%%%%%%%%%%%%%%%%%%%%%%%%%%%%%%%%%%%%%%%%%%%%%%%%%%%%%%%%%%%%%%%%%%%%%%%%%%%%%%%%%%%%
\newpage
\section{LES of a Turbulent Premixed Methane Flame}
\subsection{Model Problem}



%%%%%%%%%%%%%%%%%%%%%%%%%%%%%%%%%%%%%%%%%%%%%%%%%%%%%%%%%%%%%%%%%%%%%%%%%%%%%%%%%%%%%%%%%%%%
%-------------------------------------------------------------------------------------------
% Summary of Progress to Date and Future Work
%-------------------------------------------------------------------------------------------
%%%%%%%%%%%%%%%%%%%%%%%%%%%%%%%%%%%%%%%%%%%%%%%%%%%%%%%%%%%%%%%%%%%%%%%%%%%%%%%%%%%%%%%%%%%%
\newpage
\section{Summary of Progress to Date and Future Work}

\subsection{Progress to Date}

\begin{tabular}{|l|c|} \hline
\multicolumn{1}{|c|}{\bf{Task}} & \multicolumn{1}{|c|}{\bf{Completion Date}} \\

\hline Literature Review & September-October 2014 \\

\hline Trelis Meshing Software & November 2014 \\

\hline CFFC Group Code Flux Jacobian Analysis & December 2014 \\

\hline Trilinos Package solution for example Poisson Problem & December 2014 \\in serial and parallel configurations & \\

\hline Running a current-state LES case of a Turbulent Premixed & January 2015 \\Methane Flame using PCM-FPI to get a threshold estimation & \\ of solution run time\\

\hline
\end{tabular}

\subsection{Future Work}

\begin{tabular}{|l|c|} \hline
\multicolumn{1}{|c|}{\bf{Task}} & \multicolumn{1}{|c|}{\bf{Completion Date}} \\

\hline Implementing the approximate Adjoint Derivative to the Flux Jacobians & April 2015 \\ testing on Euler Equations  &\\

\hline Extension to Mesh adaptation & May 2015\\

\hline Application of Adjoint Problem to Navier Stokes  & June 2015\\

\hline Explicit Filters for High Order FVM implementation  & October 2015\\

\hline Conference Paper I draft & November 2015\\

\hline Coupling of High Order method with Adjoint-based AMR & December 2015\\

\hline CFD simulation of Cold Flow & January 2016\\

\hline CFD simulation of Laminar Non-Premixed Flame & February 2016\\

\hline CFD simulation of Laminar Diffusion Flame & February 2016\\

\hline Journal Paper I & April 2016\\

\hline Conference Paper II draft & April 2016\\

\hline CFD simulation of Turbulent Non-Premixed Flame & May 2016\\

\hline Journal Paper II & July 2016\\

\hline Conference Paper I Presentation  & July 2016\\

\hline Conference Paper II draft  & October 2016\\

\hline Journal Paper III & October 2016\\

\hline CFD simulation of Full Thermo-coupling & October 2016\\

\hline Conference Paper III draft  & November 2016\\

\hline Thesis write-up & September 2017 \\ 

\hline

\end{tabular}

%\newpage
\bibliographystyle{aiaa} \bibliography{journals-full,cfd,myrefs}
\end{document}
