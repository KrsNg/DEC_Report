\section{Introduction and Motivation}

Computational Fluid Dynamics (CFD) has been developed to reduce the time and cost of prototypes in fluid flow experiments. Typical product life-cycles from conception to testing involve numerous design iterations and modifications. From the mid 80s to 90s the stability and reliability of CFD algorithms was improved and this technique became more popular, and its present role has grown to include a vast spectrum of modern day industries. Gordon Moore correctly predicted that computing power would double approximately every two years.~\cite{intel:2005}. Modern day massively-paralleled computer systems have brought computing power to the peta- and exa-scale levels.\par

Real fluid flows almost always involve turbulence, and CFD methods and models have been developed to capture this phenomenon to varying extents of accuracy. Three main approaches exist: Reynolds Averaged Navier Stokes (RANS), which uses full modelling of turbulence phenomenon; Large Eddy Simulations (LES) do direct solutions of Turbulence up to some intermediate Taylor scales while applying a model to the smaller scales; Direct Numerical Simulation (DNS) performs full resolution and calculation of the turbulence without any modelling. Supercomputers have been used within academic and research lab collaborations for the Direct Numerical Simulation (DNS) of "Turbulence in a Box" such as by Kaneda and Ishihara. ~\cite{kaneda:2006}, but such calculations are limited to low Reynolds numbers and for simple cases. LES is more practical and achievable, and commonly applied to Combustion studies.\par 

This work intends to reduce the numerical solution error introduced into the simulation by the level of discretization for the governing equations, and that introduced by the mesh. The tools to be used to this end are High-Order Finite Volume Methods, and Adjoint-based Error Estimation for Mesh Adaptation. The ultimate validation will be for Thermoacoustic analysis  for a Turbomeca Model Combustor. Brisebois \textit{et al} did Thermoacoustic analysis on Experimental data for this combustor ~\cite{Brisebois:2014}. Within the combustion chamber, heat release and pressure are coupled; a sporadic fluctuations in pressure will increase the heat release, which in turn increases the pressure, thus setting up a feedback cycle. If the frequency of pressure oscillations matches the natural frequency of the chamber, the resulting resonance could lead to damage of components in the Combustion chamber, posing huge safety risks and result in very high maintenance costs.\par