% ----------------------
% THIS IS MY DEC1 REPORT
% ----------------------

\documentclass[titlepage,11pt,letterpaper]{article}

\usepackage{graphicx,epsfig,color}
\usepackage{subfigure}
\usepackage{setspace}
\usepackage{amsmath,amssymb}
\usepackage[hang,small,bf]{caption}
\usepackage[colorlinks]{hyperref}
\hypersetup{
	colorlinks=false,
	citebordercolor={0 1 0}
}
\setlength{\captionmargin}{10pt}
\setlength\parindent{0in}
\newcommand{\Matrix}[1]{\ensuremath \mathsf{#1}}
\newcommand{\Vector}[1]{\ensuremath \mathbf{#1}}
\newcommand{\Tensor}[1]{\ensuremath \vec{\vec{#1}}}
\newcommand{\Pfrac}[2]{\ensuremath \frac{\partial #1}{\partial #2}}
\newcommand{\Rfrac}[2]{\ensuremath \frac{\mbox{d} #1}{\mbox{d} #2}}
\newcommand{\Dvolume}{\ensuremath {d\Omega}}
\newcommand{\Dsurface}{\ensuremath dS}
\newcommand{\Intvolume}[1]{\ensuremath \int_\Omega {#1}\Dvolume}
\newcommand{\Intsurface}[1]{\ensuremath \int_{\Dvolume} {#1}\Dsurface}
\newcommand{\Intsurfacedot}[1]{\ensuremath \int_{\Dvolume} {#1}\cdot\Vector{\Dsurface}}
\newcommand{\Div}[1]{\ensuremath \Vector{\nabla}\cdot#1}
\newcommand{\Grad}[1]{\ensuremath \Vector{\nabla #1}}
\newcommand{\Sumedges}{\ensuremath \sum_{ik}^{\text{edges}}}
\newcommand{\nutilde}{\ensuremath \tilde{\nu}}
\newcommand{\Dx}{\ensuremath \Delta x}
\newcommand{\Dy}{\ensuremath \Delta y}
\newcommand{\Ds}{\ensuremath \Delta s}
\newcommand{\Dl}{\ensuremath \Delta l}
\newcommand{\Dt}{\ensuremath \Delta t}
\newcommand{\Tilda}[1]{\ensuremath \mbox{\~{#1}}}
\newcommand{\pseudo}{\ensuremath \textrm{\textdied}}
%\newcommand{\Fi}{\ensuremath{\mathbf{F}_{\boldsymbol{\iota}}} }
\newcommand{\Fi}{\ensuremath{\mathbf{F}} }
\newcommand{\Fv}{\ensuremath{\mathbf{F}_{\boldsymbol{\nu}}}}
%\newcommand{\Gi}{\ensuremath{\mathbf{G}_{\boldsymbol{\iota}}} }
\newcommand{\Gi}{\ensuremath{\mathbf{G}} }
\newcommand{\Gv}{\ensuremath{\mathbf{G}_{\boldsymbol{\nu}}} }
%\newcommand{\Hi}{\ensuremath{\mathbf{H}_{\boldsymbol{\iota}}} }
\newcommand{\Hi}{\ensuremath{\mathbf{H}} }
\newcommand{\Hv}{\ensuremath{\mathbf{H}_{\boldsymbol{\nu}}} }
\newcommand{\etal}{\textit{et al}}

\newcommand{\pd}[2]{\frac{\partial#1}{\partial#2}}
% Page Margins
\voffset=-1.0in
\topmargin=0.75in
\headheight=0.0in
\headsep=0.0in
\hoffset=-1.0in
\oddsidemargin=0.7in
\evensidemargin=0.7in
\bibliographystyle{unsrt}
\parindent      0.2000in

\textwidth=7.0in
\textheight=9.25in

% Line Spacing
\renewcommand{\baselinestretch}{1.33}

% Alter some LaTeX defaults for better treatment of figures:
% See p.105 of "TeX Unbound" for suggested values.
    % See pp. 199-200 of Lamport's "LaTeX" book for details.
    %   General parameters, for ALL pages:
    \renewcommand{\topfraction}{0.9}	% max fraction of floats at top
    \renewcommand{\bottomfraction}{0.8}	% max fraction of floats at bottom
    %   Parameters for TEXT pages (not float pages):
   \setcounter{topnumber}{2}
    \setcounter{bottomnumber}{2}
    \setcounter{totalnumber}{4}     % 2 may work better
    \setcounter{dbltopnumber}{2}    % for 2-column pages
    \renewcommand{\dbltopfraction}{0.9}	% fit big float above 2-col. text
    \renewcommand{\textfraction}{0.07}	% allow minimal text w. figs
    %   Parameters for FLOAT pages (not text pages):
    \renewcommand{\floatpagefraction}{0.7}	% require fuller float pages
	% N.B.: floatpagefraction MUST be less than topfraction !!
    \renewcommand{\dblfloatpagefraction}{0.7}	% require fuller float pages

    % remember to use [htp] or [htpb] for placement

%\newcommand*{\drop}{\vspace*{0.2\textheight}}



\begin{document}


\newcommand*{\titleRR}
{\begingroup
  \centering
  \vfill
  \begin{figure}
    \vspace{1.5cm}
    \begin{center}
      \includegraphics[width=0.17\textheight]{./figs/utias_logo_blue.jpg}
    \end{center}
    \vspace{1.5cm}
  \end{figure}
  {\LARGE \textbf{Adjoint-Based Error Estimation and Mesh Adaptation for LES of Turbulent Premixed Flames}}\\
  \vspace*{2.0cm}
  {\large Christopher Ngigi}\\
  \vspace*{1.cm}
  {\normalsize\itshape University of
    Toronto Institute for Aerospace Studies}\\
  {\normalsize\itshape 4925
    Dufferin Street, Toronto, Ontario, M3H 5T6, Canada}\\
  \vspace*{2.cm}
  {Doctoral Examination Committee Report} \\
  \vspace*{1.cm}
  {January 2015}
  \vfill\null
\endgroup}

\thispagestyle{empty}\titleRR
\clearpage
\setcounter{page}{1}

%\maketitle
%\tableofcontents
%\newpage

%%%%%%%%%%%%%%%%%%%%%%%%%%%%%%%%%%%%%%%%%%%%%%%%%%%%%%%%%%%%%%%%%%%%%%%%%%%%%%%%%%%%%%%%%%%%
%-------------------------------------------------------------------------------------------
% Introduction and Motivation
%-------------------------------------------------------------------------------------------
%%%%%%%%%%%%%%%%%%%%%%%%%%%%%%%%%%%%%%%%%%%%%%%%%%%%%%%%%%%%%%%%%%%%%%%%%%%%%%%%%%%%%%%%%%%%
\newpage
\section{Introduction and Motivation}

Computational Fluid Dynamics (CFD) has been developed to reduce the time and cost of prototypes in fluid flow experiments. Typical product life-cycles from conception to testing involve numerous design iterations and modifications. From the mid 80s to 90s the stability and reliability of CFD algorithms was improved and this technique became more popular, and its present role has grown to include a vast spectrum of modern day industries. Gordon Moore correctly predicted that computing power would double approximately every two years.~\cite{intel:2005}. Modern day massively-paralleled computer systems have brought computing power to the peta- and exa-scale levels.\par

Real fluid flows almost always involve turbulence, and CFD methods and models have been developed to capture this phenomenon to varying extents of accuracy. Three main approaches exist: Reynolds Averaged Navier Stokes (RANS), which uses full modelling of turbulence phenomenon; Large Eddy Simulations (LES) do direct solutions of Turbulence up to some intermediate Taylor scales while applying a model to the smaller scales; Direct Numerical Simulation (DNS) performs full resolution and calculation of the turbulence without any modelling. Supercomputers have been used within academic and research lab collaborations for the Direct Numerical Simulation (DNS) of "Turbulence in a Box" such as by Kaneda and Ishihara. ~\cite{kaneda:2006}, but such calculations are limited to low Reynolds numbers and for simple cases. LES is more practical and achievable, and commonly applied to Combustion studies.\par 

This work intends to reduce the numerical solution error introduced into the simulation by the level of discretization for the governing equations, and that introduced by the mesh. The tools to be used to this end are High-Order Finite Volume Methods, and Adjoint-based Error Estimation for Mesh Adaptation. The ultimate validation will be for Thermoacoustic analysis  for a Turbomeca Model Combustor. Brisebois \textit{et al} did Thermoacoustic analysis on Experimental data for this combustor ~\cite{Brisebois:2014}. Within the combustion chamber, heat release and pressure are coupled; a sporadic fluctuations in pressure will increase the heat release, which in turn increases the pressure, thus setting up a feedback cycle. If the frequency of pressure oscillations matches the natural frequency of the chamber, the resulting resonance could lead to damage of components in the Combustion chamber, posing huge safety risks and result in very high maintenance costs.\par

Section 2 describes the Scope of the Research while Section 3 lays out the Proposed Adaptive Mesh Refinement Strategy, including the Finite Volume Techniques. The reader will find Adjoint-Based Method for Error Estimation described in Section 4, while Sections 5 and 6 describe some of the preliminary work done to date, namely the model Poisson Problem and running an already existing LES simulation of a Turbulent Premixed Methane Flame. \par

%%%%%%%%%%%%%%%%%%%%%%%%%%%%%%%%%%%%%%%%%%%%%%%%%%%%%%%%%%%%%%%%%%%%%%%%%%%%%%%%%%%%%%%%%%%%
% Scope of Research
%%%%%%%%%%%%%%%%%%%%%%%%%%%%%%%%%%%%%%%%%%%%%%%%%%%%%%%%%%%%%%%%%%%%%%%%%%%%%%%%%%%%%%%%%%%
\section{Scope of Research}
This work proposes a two-pronged technique to reduce the numerical error arising from discretization of the governing equations and discretization of the computational domain (mesh).\par

This will involve implementation of High Order Finite Volume Scheme, as well as Adjoint-Based Error Estimation method, used to compute sensitivity of the output to a given input, and the results will be used for mesh refinement. Presently, error reduction is done by use of gradient-based error indicators used in combination with High-Order methods such as the Finite Volume Method, or Finite Element Method. to refine the mesh. More on Gradient-Based methods will be discussed in Section 4. \par

The approach to be used will be to first implement the Adjoint solver within the group CFFC Code as modified by Northrup \cite{Northrup:2013}, and validate it on already known inviscid (Euler) laminar steady solutions, and then implement it for unsteady cases, and then introduce viscosity and the temporal variation. Once the Adjoint solver works, the next step will be to advance the work done by Deconinck to test and implement appropriate LES Explicit filters \cite{Deconinck:2008} to the High-Order Finite Volume Scheme which, for Turbulent Premixed flames, uses the Favre-Averaged Navier Stokes (FANS) equations and the Central Essentially Non-Oscillatory Scheme (CENO). The reasoning behind this is that a higher order of discretization provides an improved accuracy of the solution. Combining this with Adjoint-Based Error Estimation will allow for an optimal technique for mesh refinement. The computational cost versus benefits (improved accuracy, ect) is not known at the moment - this will be a later conclusion at the end of this investigation. Ultimately, comparisons will be run to test the prediction of the pressure oscillations for High-Order and standard Second-Order schemes.\par

The CFFC Code already has most of the needed functionality required for this work. Ivan and Groth implemented the Central Essentially Non-Oscillatory Scheme \cite{ivan:2007b}, Two-Dimensional Block-Based Adaptive Mesh Refinement has been materialized by Zhang and Groth \cite{Zhang:2011}, and extended to Three Dimensions by Williamschen and Groth \cite{Williamschen:2013}.\par

The application of interest is the study of Thermoacoustic phenomenon within the model Turbomeca Combustor. This work will then attempt to compare the effect of increased accuracy (due to using Higher-Order Discretization and the Adjoint-Based Error Estimation for Mesh Refinement) on predictions of the onset of Heat and Pressure feedback cycles which can easily cause damage to the components of the combustor. Results will be compared with experimental analysis as performed by Brisebois \textit{et al} \cite{Brisebois:2014}. \par


%%%%%%%%%%%%%%%%%%%%%%%%%%%%%%%%%%%%%%%%%%%%%%%%%%%%%%%%%%%%%%%%%%%%%%%%%%%%%%%%%%%%%%%%%%%%
%-------------------------------------------------------------------------------------------
% CFD Techniques to Model Combustion
%-------------------------------------------------------------------------------------------
%%%%%%%%%%%%%%%%%%%%%%%%%%%%%%%%%%%%%%%%%%%%%%%%%%%%%%%%%%%%%%%%%%%%%%%%%%%%%%%%%%%%%%%%%%%%
\section{Proposed Adaptive Mesh Refinement Strategy}

The planned approach to perform the Adjoint-based Error Estimation (covered in Section 4) will utilize the following techniques:\par

\subsection{Large Eddy Simulation (LES)}
The level of fidelity of LES is between Direct Numerical Simulation (DNS) and Reynolds-Averaged Navier Stokes (RANS) methods. LES will apply a filter size to the turbulence scales, resolving the larger 
ones, and modelling the smaller ones (i.e. those smaller than the filter size). Due to the filtering, the Navier Stokes Equations need to be modified before being used in LES.\par  


\subsubsection{Favre-Averaged Navier Stokes Equations}
Numerical Modelling of Fluid Flow is done using governing equations of Conservation of Mass, Momentum and Energy. Since we will be modelling Turbulent Combustion, we will also include the Species Conservation Equation. Filtering is performed on the Navier-Stokes Equations, and assuming that the differentiation and filtering operations commute, we obtain the Favre-Filtered Equations \cite{HPerez:2011}:\par

Conservation of Mass:
\begin{equation}
  \label{eq:filt_mass} 
  \Pfrac{\bar{\rho}}{t} + \Pfrac{(\bar{\rho} \tilde{u}_j )}{x_j} = 0 \,,
\end{equation}
%
\indent Conservation of Momentum:
\begin{equation} 
  \label{eq:filt_momentum}
  \Pfrac{( \bar{\rho} \tilde{u}_i )}{t} 
  + \Pfrac{( \bar{\rho} \tilde{u}_i \tilde{u}_j + \delta_{ij} \bar{p} )}{x_j} 
  - \Pfrac{\check{\tau}_{ij}}{x_j} 
  =  \bar{\rho} g_i 
  + \underbrace{\Pfrac{\sigma_{ij}}{x_j}}_{\bf I}
  + \underbrace{\Pfrac{( \bar{\tau}_{ij} - \check{\tau}_{ij} )}{x_j}}_{\bf II} \,,
\end{equation}
%
\indent Conservation of Energy:
\begin{eqnarray}
  \label{eq:filt_energy}
  \Pfrac{( \bar{\rho} \tilde{E} )}{t} 
  + \Pfrac{[ (\bar{\rho} \tilde{E} + \bar{p}) \tilde{u}_j ]}{x_j}  
  - \Pfrac{( \check{\tau}_{ij} \tilde{u}_i )}{x_j}  
  + \Pfrac{\check{q}_j}{x_j} 
  & = & 
  \bar{\rho} \tilde{u}_i g_i
  - \underbrace{\Pfrac{[ \bar{\rho} (\widetilde{h_{\mathrm{s}} u_j} - \check{h}_{\mathrm{s}} \tilde{u}_j) ]}{x_j}
  }_{\bf III} 
  {} \nonumber \\  &  & {} 
  \!\!\!\!\! + \underbrace{\Pfrac{(\overline{\tau_{ij} u_i} - \check{\tau}_{ij} \tilde{u}_i)}{x_j}
  }_{\bf IV}
  - \underbrace{\Pfrac{(\bar{q}_j - \check{q}_j)}{x_j}}_{\bf V}
  {} \nonumber \\   &  & {} 
  \!\!\!\!\! - \underbrace{ \frac{1}{2} \Pfrac{[ \bar{\rho} (\widetilde{u_j u_i u_i}
      - \tilde{u}_j \widetilde{u_i u_i}) ]}{x_j}}_{\bf VI}
  {} \nonumber \\  &  & {}
  \!\!\!\!\! - \underbrace{\Pfrac{[ \sum_{\alpha=1}^N \Delta h^0_{\mathrm{f}_\alpha} \bar{\rho}
      (\widetilde{Y_\alpha u_{j}} - \tilde{Y}_\alpha \tilde{u}_j) ]}{x_j} \,,
  }_{\bf VII}
\end{eqnarray}
%
\indent Conservation of Species Fraction:
\begin{equation} 
  \label{eq:filt_species}
  \Pfrac{( \bar{\rho} \tilde{Y}_\alpha )}{t} 
  + \Pfrac{( \bar{\rho} \tilde{Y}_\alpha \tilde{u}_j )}{x_j}
  + \Pfrac{\check{\cal J}_{j,\alpha}}{x_j} 
  = - \underbrace{\Pfrac{[ \bar{\rho}(\widetilde{Y_\alpha u_j} - 
                  \tilde{Y}_\alpha \tilde{u}_j) ]}{x_j}}_{\bf VIII} 
  - \underbrace{\Pfrac{(\bar{\cal J}_{j,\alpha} - \check{\cal J}_{j,\alpha})}{x_j}}_{\bf IX} 
  + \underbrace{\bar{\dot{\omega}}_\alpha}_{\bf X} \,,
\end{equation}
%
Where we use the equation of state:
\begin{equation} 
  \label{eq:filt_eq_state} 
  \bar{p} = \bar{\rho} \check{R} \tilde{T} 
  + \underbrace{ \sum_{\alpha=1}^N R_\alpha \bar{\rho}
    (\widetilde{Y_\alpha T} - \tilde{Y}_\alpha \tilde{T}) }_{\bf XI} \,,
\end{equation}
%
%
where
%
\begin{equation}
  \label{eq:sfs_stresses}
  \sigma_{ij} = - \bar{\rho} \left( \widetilde{u_i u_j} - \tilde{u}_i \tilde{u}_j \right) \,, 
\end{equation}
%                     
is the SFS stress tensor. The total energy is written as:
%
\begin{equation}
  \tilde{E} = \check{h}_{\mathrm{s}} - \frac{\bar{p}}{\bar{\rho}} 
  + \sum_{\alpha=1}^N \Delta h^0_{\mathrm{f}_\alpha} \tilde{Y}_\alpha 
  + \frac{1}{2}\tilde{u}_i \tilde{u}_i + k_\Delta \,,
\end{equation}
%
where
% 
\begin{equation}
  k_\Delta = \frac{1}{2} \left( \widetilde{u_i u_i} - \tilde{u_i}\tilde{u_i} \right) \,,
\end{equation}
%
is the Sub-Filter Scale (SFS) Turbulent Kinetic Energy. The effects of the subfilter scales appear in the filtered total energy, $\tilde{E}$, the filtered
equation of state and the right-hand-sides of the governing continuity, momentum, energy and species mass fraction equations (i.e., terms \textbf{I},\ldots,\textbf{XI}). The symbol $(\,\check{}\,)$ is used to indicate the evaluation of expressions in terms of filtered variables, i.e.,
$\check{R}\!=\!R(\tilde{Y}_\alpha)$,
$\check{h}_{\mathrm{s}}\!=\!h_{\mathrm{s}}(\tilde{Y}_\alpha,\tilde{T})$,
and so on. The fluxes $\check{\tau}_{ij}$, $\check{q}_j$, and
$\check{\cal J}_{j,\alpha}$ are expressed as
%
\begin{eqnarray}
  \check{\tau}_{ij} & = &  2 \check{\mu} \left( \check{S}_{ij} - 
  \frac{1}{3} \delta_{ij}\check{S}_{ll} \right) \,,  \\
%
  \check{q}_j & = &  - \check{\lambda} \frac{\partial \tilde{T}}{\partial x_j} 
  - \bar{\rho} \sum_{\alpha=1}^N \check{h}_\alpha \check{D}_\alpha \frac{\partial \tilde{Y}_\alpha}{\partial x_j} \,, \\
%
  \check{\cal J}_{j,\alpha} & = &  - \bar{\rho} \check{D}_\alpha \frac{\partial \tilde{Y}_\alpha}{\partial x_j} \,,
\end{eqnarray}
%
where 
%
$\check{S}_{ij} \!=\! \frac{1}{2} \left(\partial \tilde{u}_{i}/\partial
  x_{j} + \partial \tilde{u}_j/ \partial x_i \right)$, is the strain
rate tensor calculated with the Favre-filtered velocity.  The temperature used for the molecular transport coefficients $\check{\mu}$, $\check{\lambda}$, and
$\check{D}_\alpha$ calculations is $\tilde{T}$.

\subsubsection{Explicit Filtering and Commutation Errors}
\subsubsection{Sub-Filter Scale Modelling}

\subsection{The Finite Volume Method}
This method applies a cell-centered spatial discretization to the governing equations will be discretized via the Finite Volume approach. The Favre-Averaged Navier Stokes are first described:\par
\subsubsection{The High-Order Centrally Essentially Non-Oscillatory (CENO) Scheme}

\subsection{Block-Based Adaptive Mesh Refinement (AMR)}
\subsubsection{Isotropic AMR}
\subsubsection{Anisotropic AMR}

\subsection{Combustion Modelling}
\subsubsection{Presumed Conditional Moment and Flame Propagation for Intrinsic Low Dimensional Manifold (PCM-FPI)}

%%%%%%%%%%%%%%%%%%%%%%%%%%%%%%%%%%%%%%%%%%%%%%%%%%%%%%%%%%%%%%%%%%%%%%%%%%%%%%%%%%%%%%%%%%%%
%-------------------------------------------------------------------------------------------
% Adjoint Method for Error Estimation
%-------------------------------------------------------------------------------------------
%%%%%%%%%%%%%%%%%%%%%%%%%%%%%%%%%%%%%%%%%%%%%%%%%%%%%%%%%%%%%%%%%%%%%%%%%%%%%%%%%%%%%%%%%%%%
\newpage
\section{Adjoint-Based Method for Error Estimation}
\subsection{Introduction}
\subsubsection{Steady Adjoints}
\subsubsection{Unsteady Adjoints}
\subsection{Derivation}
\subsection{Solution of Linear Systems}
\subsection{Use of Solution Error Estimates in Mesh Adaptation}
\subsection{Implementing Isotropic Mesh Refinement}


%%%%%%%%%%%%%%%%%%%%%%%%%%%%%%%%%%%%%%%%%%%%%%%%%%%%%%%%%%%%%%%%%%%%%%%%%%%%%%%%%%%%%%%%%%%%
%-------------------------------------------------------------------------------------------
% Poisson Solver Case
%-------------------------------------------------------------------------------------------
%%%%%%%%%%%%%%%%%%%%%%%%%%%%%%%%%%%%%%%%%%%%%%%%%%%%%%%%%%%%%%%%%%%%%%%%%%%%%%%%%%%%%%%%%%%%
\newpage
\section{Poisson Solver}
\subsection{Model Problem}


%%%%%%%%%%%%%%%%%%%%%%%%%%%%%%%%%%%%%%%%%%%%%%%%%%%%%%%%%%%%%%%%%%%%%%%%%%%%%%%%%%%%%%%%%%%%
%-------------------------------------------------------------------------------------------
% LES of Premixed Flames
%-------------------------------------------------------------------------------------------
%%%%%%%%%%%%%%%%%%%%%%%%%%%%%%%%%%%%%%%%%%%%%%%%%%%%%%%%%%%%%%%%%%%%%%%%%%%%%%%%%%%%%%%%%%%%
\newpage
\section{LES of a Turbulent Premixed Methane Flame}
\subsection{Model Problem}



%%%%%%%%%%%%%%%%%%%%%%%%%%%%%%%%%%%%%%%%%%%%%%%%%%%%%%%%%%%%%%%%%%%%%%%%%%%%%%%%%%%%%%%%%%%%
%-------------------------------------------------------------------------------------------
% Summary of Progress to Date and Future Work
%-------------------------------------------------------------------------------------------
%%%%%%%%%%%%%%%%%%%%%%%%%%%%%%%%%%%%%%%%%%%%%%%%%%%%%%%%%%%%%%%%%%%%%%%%%%%%%%%%%%%%%%%%%%%%
\newpage
\section{Summary of Progress to Date and Future Work}

\subsection{Progress to Date}

\begin{tabular}{|l|c|} \hline
\multicolumn{1}{|c|}{\bf{Task}} & \multicolumn{1}{|c|}{\bf{Completion Date}} \\

\hline Literature Review & September-October 2014 \\

\hline Trelis Meshing Software & November 2014 \\

\hline CFFC Group Code Flux Jacobian Analysis & December 2014 \\

\hline Trilinos Package solution for example Poisson Problem & December 2014 \\in serial and parallel configurations & \\

\hline Running a current-state LES case of a Turbulent Premixed & January 2015 \\Methane Flame using PCM-FPI to get a threshold estimation & \\ of solution run time &\\

\hline
\end{tabular}

\subsection{Future Work}

\begin{tabular}{|l|c|} \hline
\multicolumn{1}{|c|}{\bf{Task}} & \multicolumn{1}{|c|}{\bf{Completion Date}} \\

\hline Implementing the approximate Adjoint Derivative to the Flux Jacobians & April 2015 \\ testing on Euler Equations  &\\

\hline Extension to Mesh adaptation & May 2015\\

\hline Application of Adjoint Problem to Navier Stokes  & June 2015\\

\hline Explicit Filters for High Order FVM implementation  & October 2015\\

\hline Conference Paper I draft & November 2015\\

\hline Coupling of High Order method with Adjoint-based AMR & December 2015\\

\hline CFD simulation of Cold Flow & January 2016\\

\hline CFD simulation of Laminar Non-Premixed Flame & February 2016\\

\hline CFD simulation of Laminar Diffusion Flame & February 2016\\

\hline Journal Paper I & April 2016\\

\hline Conference Paper II draft & April 2016\\

\hline CFD simulation of Turbulent Non-Premixed Flame & May 2016\\

\hline Journal Paper II & July 2016\\

\hline Conference Paper I Presentation  & July 2016\\

\hline Conference Paper II draft  & October 2016\\

\hline Journal Paper III & October 2016\\

\hline CFD simulation of Full Thermo-coupling & October 2016\\

\hline Conference Paper III draft  & November 2016\\

\hline Thesis write-up & September 2017 \\ 

\hline

\end{tabular}

%\newpage
\bibliographystyle{aiaa} \bibliography{journals-full,cfd,myrefs,chris_ref}
\end{document}
