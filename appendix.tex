%%=================================================
\section{Appendix} 
\setcounter{equation}{0}
\subsection{Navier Stokes Equations} \label{section:Navier}
The conservation equations
for a thermally perfect reactive mixture of $N$ chemical species
evolving in time, $t$, and space, $\Vector{x}$, can then be written
using tensor notation as \cite{HPerez:2011}\par
%
\indent Conservation of Mass:
\begin{equation}
  \label{eq:mass}
  \frac{\partial \rho}{\partial t} 
  + \frac{\partial (\rho u_j)}{\partial x_j}  =  0 \,,
\end{equation}
%
\indent Conservation of Momentum:
\begin{equation}
  \label{eq:momentum} 
  \frac{\partial (\rho u_i)}{\partial t} 
  + \frac{\partial (\rho u_i u_j + \delta_{ij} p)}{\partial x_j}  
  - \frac{\partial \tau_{ij}}{\partial x_j}
  =  \rho g_i \,, 
\end{equation}
%
\indent Conservation of Energy:
\begin{equation}
  \label{eq:energy}
  \frac{\partial (\rho E)}{\partial t} 
  + \frac{\partial [(\rho E + p) u_j]}{\partial x_j} 
  - \frac{\partial (\tau_{ij} u_i)}{\partial x_j} 
  + \frac{\partial q_j}{\partial x_j}
  =  \rho g_i u_i \,,  
\end{equation}
%
\indent Conservation of Species Fraction:
\begin{equation}
  \label{eq:species}
  \frac{\partial (\rho Y_\alpha)}{\partial t} 
  + \frac{\partial (\rho Y_\alpha u_j)}{\partial x_j}
  + \frac{\partial \mathcal{J}_{j,\alpha} }{\partial{x_j}} 
  =  \dot{\omega}_{\alpha} \,, 
\end{equation}
%
where
%
\begin{eqnarray}
  \label{eq:stress_tensor}
  \tau_{ij} & = & \mu \left( \frac{\partial u_i}{\partial x_j} 
  + \frac{\partial u_j}{\partial x_i} \right) 
  - \frac{2}{3} \mu \delta_{ij} \frac{\partial u_l}{\partial x_l} \,,  \\
  \label{eq:heat_flux}
  q_j & = & - \lambda \frac{\partial T}{\partial x_j} 
  - \rho \sum_{\alpha=1}^N h_\alpha\mathcal{D}_\alpha \frac{\partial Y_\alpha}{\partial x_j}  \,, \\
  \label{eq:species_diffusive_flux}
  \mathcal{J}_{j,\alpha} & = & - \rho \mathcal{D}_\alpha \frac{\partial Y_\alpha}{\partial x_j} \,,  
\end{eqnarray}
%



%%===================================================
\subsection{Favre-Averaged Navier Stokes Equations} \label{section:Favre}
Favre-Filtering, essentially a density-weighted filtering procedure is defined as:
\begin{equation}
 \tilde{\phi}=\frac{\overline{\rho \phi}}{\overline{\rho}},
 \end{equation}
where $\tilde{\phi}$ represents the Favre-filtered variable.

Performing this on the governing equations, and assuming that the differentiation and filtering operations commute, we obtain the Favre-Filtered Equations as described by H. Perez \cite{HPerez:2011} as follows:\par

Conservation of Mass:
\begin{equation}
  \label{eq:filt_mass} 
  \Pfrac{\bar{\rho}}{t} + \Pfrac{(\bar{\rho} \tilde{u}_j )}{x_j} = 0 \,,
\end{equation}
%
\indent Conservation of Momentum:
\begin{equation} 
  \label{eq:filt_momentum}
  \Pfrac{( \bar{\rho} \tilde{u}_i )}{t} 
  + \Pfrac{( \bar{\rho} \tilde{u}_i \tilde{u}_j + \delta_{ij} \bar{p} )}{x_j} 
  - \Pfrac{\check{\tau}_{ij}}{x_j} 
  =  \bar{\rho} g_i 
  + \underbrace{\Pfrac{\sigma_{ij}}{x_j}}_{\bf I}
  + \underbrace{\Pfrac{( \bar{\tau}_{ij} - \check{\tau}_{ij} )}{x_j}}_{\bf II} \,,
\end{equation}
%
\indent Conservation of Energy:
\begin{eqnarray}
  \label{eq:filt_energy}
  \Pfrac{( \bar{\rho} \tilde{E} )}{t} 
  + \Pfrac{[ (\bar{\rho} \tilde{E} + \bar{p}) \tilde{u}_j ]}{x_j}  
  - \Pfrac{( \check{\tau}_{ij} \tilde{u}_i )}{x_j}  
  + \Pfrac{\check{q}_j}{x_j} 
  & = & 
  \bar{\rho} \tilde{u}_i g_i
  - \underbrace{\Pfrac{[ \bar{\rho} (\widetilde{h_{\mathrm{s}} u_j} - \check{h}_{\mathrm{s}} \tilde{u}_j) ]}{x_j}
  }_{\bf III} 
  {} \nonumber \\  &  & {} 
  \!\!\!\!\! + \underbrace{\Pfrac{(\overline{\tau_{ij} u_i} - \check{\tau}_{ij} \tilde{u}_i)}{x_j}
  }_{\bf IV}
  - \underbrace{\Pfrac{(\bar{q}_j - \check{q}_j)}{x_j}}_{\bf V}
  {} \nonumber \\   &  & {} 
  \!\!\!\!\! - \underbrace{ \frac{1}{2} \Pfrac{[ \bar{\rho} (\widetilde{u_j u_i u_i}
      - \tilde{u}_j \widetilde{u_i u_i}) ]}{x_j}}_{\bf VI}
  {} \nonumber \\  &  & {}
  \!\!\!\!\! - \underbrace{\Pfrac{[ \sum_{\alpha=1}^N \Delta h^0_{\mathrm{f}_\alpha} \bar{\rho}
      (\widetilde{Y_\alpha u_{j}} - \tilde{Y}_\alpha \tilde{u}_j) ]}{x_j} \,,
  }_{\bf VII}
\end{eqnarray}
%
\indent Conservation of Species Fraction:
\begin{equation} 
  \label{eq:filt_species}
  \Pfrac{( \bar{\rho} \tilde{Y}_\alpha )}{t} 
  + \Pfrac{( \bar{\rho} \tilde{Y}_\alpha \tilde{u}_j )}{x_j}
  + \Pfrac{\check{\cal J}_{j,\alpha}}{x_j} 
  = - \underbrace{\Pfrac{[ \bar{\rho}(\widetilde{Y_\alpha u_j} - 
                  \tilde{Y}_\alpha \tilde{u}_j) ]}{x_j}}_{\bf VIII} 
  - \underbrace{\Pfrac{(\bar{\cal J}_{j,\alpha} - \check{\cal J}_{j,\alpha})}{x_j}}_{\bf IX} 
  + \underbrace{\bar{\dot{\omega}}_\alpha}_{\bf X} \,,
\end{equation}
%
Where we use the equation of state:
\begin{equation} 
  \label{eq:filt_eq_state} 
  \bar{p} = \bar{\rho} \check{R} \tilde{T} 
  + \underbrace{ \sum_{\alpha=1}^N R_\alpha \bar{\rho}
    (\widetilde{Y_\alpha T} - \tilde{Y}_\alpha \tilde{T}) }_{\bf XI} \,,
\end{equation}
%
%
where
%
\begin{equation}
  \label{eq:sfs_stresses}
  \sigma_{ij} = - \bar{\rho} \left( \widetilde{u_i u_j} - \tilde{u}_i \tilde{u}_j \right) \,, 
\end{equation}
%                     
is the SFS stress tensor. The total energy is written as:
%
\begin{equation}
  \tilde{E} = \check{h}_{\mathrm{s}} - \frac{\bar{p}}{\bar{\rho}} 
  + \sum_{\alpha=1}^N \Delta h^0_{\mathrm{f}_\alpha} \tilde{Y}_\alpha 
  + \frac{1}{2}\tilde{u}_i \tilde{u}_i + k_\Delta \,,
\end{equation}
%
where
% 
\begin{equation}
  k_\Delta = \frac{1}{2} \left( \widetilde{u_i u_i} - \tilde{u_i}\tilde{u_i} \right) \,,
\end{equation}
%
is the Sub-Filter Scale (SFS) Turbulent Kinetic Energy. The effects of the subfilter scales appear in the filtered total energy, $\tilde{E}$, the filtered
equation of state and the right-hand-sides of the governing continuity, momentum, energy and species mass fraction equations (i.e., terms \textbf{I},\ldots,\textbf{XI}). The symbol $(\,\check{}\,)$ is used to indicate the evaluation of expressions in terms of filtered variables, i.e.,
$\check{R}\!=\!R(\tilde{Y}_\alpha)$,
$\check{h}_{\mathrm{s}}\!=\!h_{\mathrm{s}}(\tilde{Y}_\alpha,\tilde{T})$,
and so on. The fluxes $\check{\tau}_{ij}$, $\check{q}_j$, and
$\check{\cal J}_{j,\alpha}$ are expressed as
%
\begin{eqnarray}
  \check{\tau}_{ij} & = &  2 \check{\mu} \left( \check{S}_{ij} - 
  \frac{1}{3} \delta_{ij}\check{S}_{ll} \right) \,,  \\
%
  \check{q}_j & = &  - \check{\lambda} \frac{\partial \tilde{T}}{\partial x_j} 
  - \bar{\rho} \sum_{\alpha=1}^N \check{h}_\alpha \check{D}_\alpha \frac{\partial \tilde{Y}_\alpha}{\partial x_j} \,, \\
%
  \check{\cal J}_{j,\alpha} & = &  - \bar{\rho} \check{D}_\alpha \frac{\partial \tilde{Y}_\alpha}{\partial x_j} \,,
\end{eqnarray}
%
where 
%
$\check{S}_{ij} \!=\! \frac{1}{2} \left(\partial \tilde{u}_{i}/\partial
  x_{j} + \partial \tilde{u}_j/ \partial x_i \right)$, is the strain rate tensor calculated with the Favre-filtered velocity.  The temperature used for the molecular transport coefficients $\check{\mu}$, $\check{\lambda}$, and $\check{D}_\alpha$ calculations is $\tilde{T}$.

Modelling of the SFS terms is required to close the above system of equations. Term \textbf{II} is neglected under the assumption that the filtered viscous stresses, $\bar{\tau}_{ij}$, can be approximated to the viscous stresses evaluated in terms of the Favre-filtered velocity, $\check{\tau}_{ij}$. Following similar assumptions for the total heat and species mass diffusion fluxes, terms \textbf{V} and
\textbf{IX} may be neglected. Vreman \etal ~\cite{vreman:1995} performed a priori LES of a mixing layer at different Mach numbers and concluded that neglecting the non-linearities of the diffusion terms in the momentum and energy equations is acceptable. Regarding term \textbf{IV} (SFS viscous diffusion), it is generally much smaller than the other terms that require a model~\cite{martin:2000}, and so is neglected. As for term \textbf{XI} (SFS temperature-species correlation), it is assumed to be small and generally neglected.
%
Following the work of Knight \etal.~\cite{knight:1998}, term \textbf{VI} (the SFS turbulent diffusion) can be modelled in terms of the SFS stresses and the resolved velocity as
%
\begin{equation}
  - \frac{\bar{\rho} \left(\widetilde{u_j u_i u_i} - \tilde{u}_j\widetilde{u_i u_i} \right)}{2} 
  = \sigma_{ij}\tilde{u}_i \,.
\end{equation}
%
Term \textbf{VII} involves the SFS species fluxes (term \textbf{VIII}) and is closed with the modelled SFS species fluxes. 