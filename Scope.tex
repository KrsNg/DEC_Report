\section{Scope of research}

\begin{itemize}

\item The goal is Reducing numerical error using high order CENO and adjoint based error estimation, $\mathcal{O}(h^p)$ $\rightarrow$ $h$ and $p$ adaptation

\item This work proposes a two-pronged technique to reduce the numerical error within the discretized governing equations, as well as in the computational domain (mesh).  Implementation of high order finite volume method (FVM) as well as adjoint-based error estimation is believed to be a significant step to achieve this. The adjoint technique can be used for $\mathcal{O}(h^p)$ refinement: evaluating error estimates and then locally switching between h and p refinement, based on the computational cost savings.

\item The CFFC code already has most of the needed formulations required for this work. 
\begin{itemize}
\item Combustion modeling:
  \begin{itemize}
   \item PCM-FPI: This model allows detailed chemical kinetics via tabulation of precomputed laminar premixed flames, and has already been implemented within the CFFC code by H. Perez (2011) ~\cite{hperez:2011a}
   \end{itemize}  

\item Favre-filtered Navier Stokes governing equations: The FANS formulation has already been implemented within the group CFFC code.
      
\item Large Eddy Simulation:
  \begin{itemize}
   \item Explicit filtering work done by Deconinck, ~\cite{Deconinck:2008}
   \item Sub-filter scale (SFS) modeling work done by H. Perez ~\cite{hperez:2011a}
  \end{itemize}

\item High-order finite volume methods: CENO technique - benefits of higher accuracy on a coarse mesh was implemented by Ivan and Groth ~\cite{ivan:2007, ivan:2013b}

\item AMR
   \begin{itemize}
   \item Block-based AMR: speed and parallelization [Groth et al 1999]
   \item Anisotropic AMR reduces cell count (computational cost): 2-D initial work by Zhang and Groth \cite{zhang:2011b}, and extended to 3-D by Williamschen and Groth \cite{Williamschen:2013}
   \item Freret and Groth, \cite{Freret:2015} extended the work of Williamschen to include a non-uniform block formulation. This increases code accuracy while saving computational resources.
   \end{itemize}
   
\item Solution method: Northrup and Groth~\cite{Northrup:2013, northrup:2013a} implemented an implicit GMRES solver that improves the robustness and capability of the CFFC code by relieving it from the Courant-Friedric-Levy (CFL) condition constraints.  

\end{itemize}

\item Workflow: 
\begin{itemize}
	\item Creating a framework for the adjoint based error estimation method. The approach to be used will be to first formulate the adjoint solver within the group CFFC Code. 
	\item Once the adjoint is formulated, it will be validated on already known inviscid (Euler) laminar steady solutions, and then implemented for unsteady cases. Complexity will be introduced by adding viscosity and then a temporal variation. 
	\item The next step will be to advance the work done by Deconinck on explicit filters and implement them \cite{Deconinck:2008} to the high-order FVM scheme. 
\end{itemize}

\item The implementation of the adjoint based error estimation would allow for significant reductions in numerical error, and these will be accounted for in the duration of this research.
\end{itemize}

