\subsection{Combustion Modelling via the Presumed Conditional Moment and Flame Propagation for Intrinsic Low Dimensional Manifold (PCM-FPI) Model}

When performing LES of turbulent reactive flows, there is need to predict the chemical reaction rates. Implementation of the PCM-FPI model in LES within the CFD Group was carried out by Hern{\'a}ndez-P{\'e}rez ~\cite{HPerez:2011} and Shahbazian \etal ~\cite{Shahbazian:2011}.\par

The FPI model was proposed to build databases based on detailed simulations of simple flames. The basis of tabulations is the steady-state one-dimensional laminar flame, which are solved using CANTERA software. These flame quanitities are stored in a table that is then read by the CFFC program. The main objective of the FPI tabulation technique is to reduce the cost of performing reactive
flow computations with large detailed chemical kinetic mechanisms, but still retain the accuracy of detailed results, by building databases of relevant quantities based on detailed simulations of simple flames. ~\cite{HPerez:2011} \par

The PCM technique uses a statistical approach and utilizes Probability Density Functions (PDFs). The approach presumes a PDF-like distribution of a subfilter-scale fluctuating quantity. The PDF subfilter is incorporated into the Favre-filtered reaction rates for species.\par

Combining PCM with FPI allows an approach to model complex chemistry from tabulated data, is a very general model that can be used for premixed, non-premixed and partially-premixed flames. e.g. In the event that turbulent premixed combustion is the focus of the analysis, look up tables of filtered terms are built up from laminar premixed flamelet data.\par
