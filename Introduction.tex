\section{Introduction and motivation}

The development of computational fluid dynamics (CFD) over the past 30-40 years has been carried out with a goal of helping to reduce the time and cost of prototype design for a wide range of engineering systems. Practical reactive flows almost always involve turbulence, and CFD methods have been developed to help capture such complex phenomena to varying extents of accuracy. The corresponding physical experiments can, in many cases, take a long time and be very expensive to conduct. Three main numerical approaches exist for the treatment of turbulent flows. Reynolds averaged Navier Stokes (RANS) uses full modeling of the turbulent scales; Large eddy simulations (LES) model sub-filter scales (SFS) while resolving larger scales; and Direct numerical simulations (DNS) resolve all the turbulent scales. Supercomputers have been used within academic and research lab collaborations for the DNS of canonical cases such as turbulence in a box as considered by Kaneda and Ishihara~\cite{kaneda:2006}, but such calculations are very expensive and limited to simple cases with low Reynolds numbers. More complicated analyses such as those involving combustion simulations use LES to achieve an overall good prediction of results at an accuracy greater than RANS, but at a cost lower than DNS.\par 

To improve the accuracy of LES, we need to reduce and control the error within the simulation. Key sources of error in CFD include numerical and modeling errors. Modeling errors in LES can be significantly reduced by using explicit discrete commutative filters. Numerical errors can be reduced by applying mesh refinement ($h$ refinement) or by increasing the polynomial discretization of the numerical scheme ($p$ refinement). High order discretizations, generally classified as those having order $p>2$, have less numerical dissipation and require fewer mesh points to achieve solution accuracy (as compared to a $2^{nd}$ order method, for example). To minimize the computational costs, refinement would be \textit{localized} i.e.\ applied specifically to regions of high error. Ways to quantify the error include the gradient (physics)-based approach, but this technique does not guarantee error reduction for continual levels of mesh refinement($h$ refinement).  (More on gradient-based methods will be discussed in Section \ref{section:Adjoint}). A more effective approach to derive a set of defined error estimates is the adjoint-based method, where localized numerical errors are obtained at a potentially lower cost. This work proposes the combination of $h$ refinement with $p$ refinement within an adjoint-based error estimation framework. That is, the additional use of high-order discretization schemes  to further improve accuracy of the CFD results, while lowering the cost to attain them. This combination, expressed as an adjoint-based $\mathcal{O}(h^p)$ error estimation technique, will allow for better assessment and control of error.\par 