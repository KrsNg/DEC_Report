\section{Scope of Research}
This work proposes a two-pronged technique to reduce the numerical error arising from discretization of both the governing equations and the computational domain (mesh) involving implementation of high order finite volume method (FVM) as well as adjoint-based error estimation to compute sensitivity of the output to a change in a given input. The results will be later used for mesh refinement criteria instead of a physics or gradient-based evaluation. \par

The approach to be used will be to first implement the adjoint solver within the group CFFC Code as modified by Northrup \cite{Northrup:2013}, and validate it on already known inviscid (Euler) laminar steady solutions, and then implement it for unsteady cases, and then introduce viscosity and the temporal variation. Once the Adjoint solver works, the next step will be to advance the work done by Deconinck to test and implement appropriate LES Explicit filters \cite{Deconinck:2008} to the high-order FVM which, for simulation of reacting flows, uses the Favre-averaged Navier Stokes (FANS) equations and the central essentially non-oscillatory Scheme (CENO). The application of interest is the improved simulation of turbulent premixed flames while reducing the computational costs. A benefits analysis will then be performed upon conclusion. \par

The CFFC code already has most of the needed formulations required for this work. Ivan and Groth implemented CENO [\cite{ivan:2007, ivan:2013b}], 2-D block-based adaptive mesh refinement (AMR) has been materialized by Zhang and Groth \cite{zhang:2011b}, and extended to 3-D by Williamschen and Groth \cite{Williamschen:2013}, as well as Freret and Groth \cite{Freret:2015}. Northrup and Groth \cite{Northrup:2013} implemented an iterative GMRES solver for CFFC which relieves the code from the Courant-Friedric-Levy (CFL) condition constraints.\par


